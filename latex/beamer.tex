\documentclass{beamer}

\title{Haskell}
\author{Lee}
%\usetheme{Berlin}
\usetheme{Warsaw}
%\usetheme{Berkeley}
%\usecolortheme{beaver}
\begin{document}
\begin{frame}
  \maketitle
\end{frame}

\begin{frame}{Outline}
  \tableofcontents
\end{frame}

\section{Part 1}

\begin{frame}[t]{Slide 1}
  \begin{itemize}
  \item<1-> Iterm 1
  \item<2-> Iterm 2
  \item<3-> Iterm 3
  \end{itemize}
  \begin{block}<4->{Theorem}
    The theorem of Pythagoras can be written as
  $$ x^2 + y^2 = z^2$$
  \end{block}
\end{frame}

\section{Part 2}
\begin{frame}{Slide 2}
  \begin{columns}
    \begin{column}{0.5\textwidth}
      This is logo
    \end{column}
    \begin{column}{0.5\textwidth}
  \begin{center}
    \includegraphics[width=0.7\textwidth]{logo.png}
  \end{center}
    \end{column}
  \end{columns}
\end{frame}
\end{document}
