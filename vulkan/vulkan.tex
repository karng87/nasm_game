\documentclass[a4paper]{moderncv}
\usepackage[scale=0.9]{geometry}
\usepackage{blindtext}
\usepackage{amsmath, kotex}
\usepackage[utf8]{inputenc}
\usepackage[T1]{fontenc}

\moderncvtheme[blue]{classic}

\firstname{Vulkan}
\familyname{Tutorial}
\title{vulkan}

\address{Seoul, Korea}
\mobile{010-3201-xxxx}
\email{karng87@gmail.com}
%\photo{logo.png}

\recipient{수령지}{Songpa ku\\Seoul\\Korea}
\opening{latex 에게}
\closing{안녕~}

\begin{document}
%\makelettertitle
%\blindtext
%\makeletterclosing
%\newpage

\maketitle
\cventry{vkSurface(WSI)}{$\leftarrow$ PresentQueue}{$\leftarrow$ Swapchain Images}{$\leftarrow$ Swapchain ImageView}{$\leftarrow$ Framebuffer}{}
\section{window surface}
\subsection{WSI}
\cvitem{}{use WSI(window system intergration)}
\cvlistitem{VK\_KHR\_surface}
\cvlistitem{VK\_KHR\_win32\_surface}
\cvlistitem{VK\_KHR\_xcb\_surface}
\cvlistitem{VK\_KHR\_android\_surface}
\section{present Queue}
\cventry{image}{make renderring}{}{}{}{}
\cventry{check by}{vkGEtPhysicalDeviceSurfaceSupportKHR()}{}{}{}{}
\cventry{submit}{present queque}{from rederred image}{}{}{}
\subsection{graphics queue family}
\section{Swapchain}
\cvlistitem{출력장치에 연속으로 표시 할 수 있는 하나 이상의 image collection}
\cvlistitem{Refresh rate sync 에 맞춰서 rendering 된 결과들을 화면에 출력하는 역할}
\cvlistitem{Swapchain 에 연결된 image 에게 graphics queue 에 기록된 operation 들로 rendering 을 수행하고 present queue 에 submit}
\cventry{swapchain image}{image resource}{swapchain 에 연결되는 실체}{}{}{}
\cventry{swapchain image view}{meta data about swapchain}{ex) RGBA component}{surface format}{mipmap}{image array}
\subsection{querying for swapchain support}
\cventry{}{swapchain 을 만들기 위해서 필요한 것들}{}{}{}{}
\cventry{}{vkGetPhysicalDeviceSurface\_KHR}{gpu,surface}{}{}{}
\cvlistitem{Surface Capabilities}
\cvlistitem{Surface Format}
\cvlistitem{Surface Present Modes}
\subsection{Surface Capabilities}
\cvlistitem{capabilities 값들중 extent값을 사용}
\subsection{Presenttaion Mode}
\cvlistitem{화면에 이미지를 표하는 조건을 설정}
\cventry{}{VK\_PRESENT\_MODE\_FOIFO\_KHR}{wait when queue full}{}{}{}
\cventry{}{VK\_PRESENT\_MODE\_FOIFO\_RELAXED\_KHR}{no wait when queue empty}{}{}{}
\cventry{}{VK\_PRESENT\_MODE\_MAILBOX\_KHR}{no wait when queue full}{}{}{}

\section{Framebuffer}
\cventry{:=}{Color,depth,stencil target 이 되는 buffer}{}{}{}{}
\cventry{}{each swapchain image view 를 연결하여 frame buffer 를 생성}{}{}{}{}
\cventry{how to create}{swapchain image view(color,detpth,stencil imag view) + rneder pass + swapchain extent, number of swapchains}{}{}{}{}
\section{Terms}
\cventry{}{agnostic}{중립적}{}{}{}
\end{document}
