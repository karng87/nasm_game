\documentclass[a4paper]{moderncv}
\usepackage[scale=0.9]{geometry}
\usepackage{amsmath, kotex}
\usepackage[T1]{fontenc}
\usepackage[utf8]{inputenc}

\moderncvtheme[blue]{classic}
\firstname{Type}
\familyname{Haskell}
\title{functional progrmming}
\address{Seoul, Korea}
\mobile{010-3201-xxxx}
\email{karng87@gmail.com}
\photo{logo.png}
\recipient{수령지}{Building Type\\SongPaku\\Seoul\\Korea}
\opening{Type 에게}
\closing{안녕~}

\begin{document}
%\makelettertitle
%\blindtext
%\makeletterclosing
%\newpage

\maketitle

\section{Type}
\cvitem{type $\cong$ set}{반드시 정해진 규격으로 만들어서 분류 할 수 있게 하는 개념}
\cvitem{}{type Id parameters = TypeConstructor Id parameters ...}
\cvitem{}{data Id parameters = TypeConstructor Id parameters ...}
\cvitem{type}{type ReadS a = String -> [(a,String)]}
\cvitem{data}{data Set a = NilSet | ConsSet a(Set a)}

\section{Type syntax}
\cvitem{Capital Letter}{Id, Constructors }
\cvitem{keyword}{type, data}
\cvitem{Id}{tpye name, type identifire}
\cvitem{parameters}{$\forall a \in $ an arbitary Type}
\cvitem{}{constrain the category}
\cvitem{Values}{$\cong$ the type Id of elements}

\section{type Id parameters = TypeConstructor Id parameters $\mid$ ...}
\cvitem{type}{기존(built in) 타입(집합)의 원소만을 이용하여 새로운 타입(집합)을  정의한다.}
\section{data Id parameters = TypeConstructor Id parameters $\mid$ ...}
\cvitem{data}{완전히 새로운 원소를 정의하여 새로운 타입(집합)을 정의한다.}

\section{Category}
\cvitem{category $\cong$ class}{공통적인 성질을 가진 것들을 모아놓은것}

\section{class, instance}
\cvitem{class}{규격화된 타입을 모아서 통합 할 수 있게 하는 개념}
\cvitem{}{isomorphism}
\cvitem{instance}{실제로 통합시켜야 하는 구체적인 타입을 모아놓고 구현하는 개념}

\section{parameter, argument}
\cvitem{parameter}{부속 건물- 재료를 제공해주는 역할}
\cvitem{argument}{실제 사용되는 곳에서 parameter가 구체적으로 확정 되는 경우}

\cvlistitem{types are how you describe the data your program will work with.}
\cvitem{$data [contex \implies]$}{$
  \underbrace{type}_{name}\; 
  \underbrace{tv_1 \cdots tv_i}_{parameters} 
  = \underbrace{cons_1}_{constructor} \; 
  \underbrace{cons_1^{p_1} \cdots cons_1^{p_n}}_{consructor\; parameter} 
  \mid \cdots \mid con_m \; c_mt_1 \cdots c_mt_q \; 
  [deriving]$}

\cvitem{[contex $\implies$]}{optional context}

\section{kind}
\cvitem{kind}{how many the type have parameters}
\cvitem{}{$* \to * \to *$}

\section{Terms}
\cvlistitem{https://wiki.haskell.org/Keywords}
\cvitem{::}{Read as ``has type"}
\cvitem{;}{Statement separator}
\cvitem{,}{Separator in lists, tuples, records.}
\cvitem{=}{Used in definitions}
\cvitem{=>}{Used to indicate instance contexts.}
\cvitem{kind}{In type theory, a kind is the type of a type constructor,}
\cvitem{}{or the type of a higher-order type operator.}
\cvitem{}{* $\iff$ the kind of any data type - called type.}
\cvitem{\_}{wild card}
\cvitem{infix function}{x + y}
\cvitem{}{to prefix $\iff$ (+) x  y}
\cvitem{prefix function}{add x y}
\cvitem{}{to infix $\iff$ x \`{}add\`{} y}

\cvitem{Perfect number}{a positive integer is perfect if it equals the sum of all of its factors, excluding the number itself}
\cvitem{induction}{귀납, 유도}
\cvitem{congruence}{합동}
\cvitem{conjugate}{켤레}
\cvitem{Terminal}{grammer에 종속되는 symbol}
\cvitem{Token}{= Terminal symbol}
\cvitem{Nonterminal}{grammer에 독립적인 symbol}
\cvitem{Start Symbol}{Nonterminal symbol로 시작해야 한다.}

\end{document}

