\documentclass[a4paper]{moderncv}
\usepackage[scale=0.9]{geometry}
\usepackage{amsmath, kotex}
\usepackage[T1]{fontenc}
\usepackage[utf8]{inputenc}

\moderncvtheme[blue]{classic}
\firstname{Type}
\familyname{Haskell}
\title{functional progrmming}
\address{Seoul, Korea}
\mobile{010-3201-xxxx}
\email{karng87@gmail.com}
\photo{logo.png}
\recipient{수령지}{Building Type\\SongPaku\\Seoul\\Korea}
\opening{Type 에게}
\closing{안녕~}

\begin{document}
%\makelettertitle
%\blindtext
%\makeletterclosing
%\newpage

\maketitle

\section{type, constructor}
\cvitem{type}{반드시 정해진 규격으로 만들어서 분류 할 수 있게 하는 개념}
\cvitem{}{set}
\cvitem{}{data}

\cvitem{constructor}{타입 규격을 정해주고 하나의 규격을 선택 하는 방법을 제공}
\cvitem{}{elements}

\section{class, instance}
\cvitem{class}{규격화된 타입을 모아서 통합 할 수 있게 하는 개념}
\cvitem{}{isomorphism}
\cvitem{instance}{실제로 통합시켜야 하는 구체적인 타입을 모아놓고 규현하는 개념}

\section{parameter, argument}
\cvitem{parameter}{부속 건물- 재료를 제공해주는 역할}
\cvitem{argument}{실제 사용되는 곳에서 parameter가 구체적으로 확정 되는 경우}

\cvlistitem{types are how you describe the data your program will work with.}
\cvitem{$data [contex \implies]$}{$
  \underbrace{type}_{name}\; 
  \underbrace{tv_1 \cdots tv_i}_{parameters} 
  = \underbrace{cons_1}_{constructor} \; 
  \underbrace{cons_1^{p_1} \cdots cons_1^{p_n}}_{consructor\; parameter} 
  \mid \cdots \mid con_m \; c_mt_1 \cdots c_mt_q \; 
  [deriving]$}

\cvitem{[contex $\implies$]}{optional context}
\end{document}

