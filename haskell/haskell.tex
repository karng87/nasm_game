\documentclass[a4paper,12pt]{article}
% latin modern
%\usepackage{lmodern}
%\usepackage[fontsize=16]{fontsize}

\usepackage{listings}
\usepackage{amsmath, amsfonts, amsthm, xcolor, kotex}
\theoremstyle{definition}
\newtheorem{definition}{Definition}[section]

\usepackage{indentfirst}
  \setlength\parindent{2em}


\begin{document}
\section{Cardinality} Element numbers
  \begin{equation}
    ^{Enum}_{Read}{\color{red}{C}}^{Num}_{Char}
  \end{equation}

\section{Type} A type is a name for a collection of related values. For example, in Haskell the basic type

\subsection{Type inference} Every welled formed expression has a type, which can be automatically calculated at compile time using a process called type inference.

\subsection*{Type Errors} Applying a function to arguments of the wrong type is called a type error. 
\lstinline!1 + False!

\subsection{Basic Type}
\begin{lstlisting}
  Bool  - logical values 
  Char  - single characters 
  String 
  Num Int Float Double
  List  - A list is sequence of values of the same type
  Tuple - A tuple is a sequence of values of different types
    (False, True) :: (Bool, Bool)
    (False, 'a', True) :: (Bool, Char, Bool)
    ('a', (False, 'b')) :: (Char, (Bool, Char)
    (True, ['a', 'b']) :: (Bool, [Char]) 
  Function Types  - mapping from values of one type to values of another type
    not :: Bool -> Bool
    eve :: Int -> Bool
  ->  - is typed at the keyboard as ->.
\end{lstlisting}

\subsection{:Type} \lstinline!:type! This command calculates the types of an expression, without evaluating it
\begin{lstlisting}
  not False 
  >True 
  :type not False 
  >not False :: Bool
\end{lstlisting}

\section{List} [ ], :, [x|x<-xs,..]

\subsection{[ ]} empty list 

\subsection*{:} cons(construct)

\subsection*{++} get the nth element of a list 

\subsection*{!!}

append two lists

\subsection*{` `}

infix 
$x$ `$f$` $y$ is just syntactic sugar for $f x y$. 

\subsection*{$\mid$} comprehension(list comprehension) 
$[a \mid a \leftarrow xs, a \leq x] \\ \{ a \mid a \in xs, a \leq x\}$

\subsection*{head} get the first element of a list

\subsection*{init} remove the last element of a list

\subsection*{tail} get the rest of a list except first element

\subsection*{take n} get the first n elements of a list 

\subsection*{drop n} get nth the last elements of a list 

\subsection*{length} get the length of a list 

\subsection*{sum} get the sum of a list of numbers

\subsection*{product} get the product of a list of numbers

\subsection*{reverse} get the reverse of a list of numbers

\section{Curried Function} Functions with multiple arguments are also possible by returning functions as result:
  \begin{lstlisting}
    add' :: Int -> (Int -> Int)
    add' x y = x+y
  \end{lstlisting}
\end{document}
